% Generated by Sphinx.
\def\sphinxdocclass{report}
\documentclass[letterpaper,10pt,english]{sphinxmanual}
\usepackage[utf8]{inputenc}
\DeclareUnicodeCharacter{00A0}{\nobreakspace}
\usepackage{cmap}
\usepackage[T1]{fontenc}
\usepackage{babel}
\usepackage{times}
\usepackage[Bjarne]{fncychap}
\usepackage{longtable}
\usepackage{sphinx}
\usepackage{multirow}

\addto\captionsenglish{\renewcommand{\figurename}{Fig. }}
\addto\captionsenglish{\renewcommand{\tablename}{Table }}
\floatname{literal-block}{Listing }



\title{ENR835 Documentation}
\date{February 03, 2016}
\release{1.0.0}
\author{Antoine}
\newcommand{\sphinxlogo}{}
\renewcommand{\releasename}{Release}
\makeindex

\makeatletter
\def\PYG@reset{\let\PYG@it=\relax \let\PYG@bf=\relax%
    \let\PYG@ul=\relax \let\PYG@tc=\relax%
    \let\PYG@bc=\relax \let\PYG@ff=\relax}
\def\PYG@tok#1{\csname PYG@tok@#1\endcsname}
\def\PYG@toks#1+{\ifx\relax#1\empty\else%
    \PYG@tok{#1}\expandafter\PYG@toks\fi}
\def\PYG@do#1{\PYG@bc{\PYG@tc{\PYG@ul{%
    \PYG@it{\PYG@bf{\PYG@ff{#1}}}}}}}
\def\PYG#1#2{\PYG@reset\PYG@toks#1+\relax+\PYG@do{#2}}

\expandafter\def\csname PYG@tok@mh\endcsname{\def\PYG@tc##1{\textcolor[rgb]{0.13,0.50,0.31}{##1}}}
\expandafter\def\csname PYG@tok@si\endcsname{\let\PYG@it=\textit\def\PYG@tc##1{\textcolor[rgb]{0.44,0.63,0.82}{##1}}}
\expandafter\def\csname PYG@tok@kp\endcsname{\def\PYG@tc##1{\textcolor[rgb]{0.00,0.44,0.13}{##1}}}
\expandafter\def\csname PYG@tok@mo\endcsname{\def\PYG@tc##1{\textcolor[rgb]{0.13,0.50,0.31}{##1}}}
\expandafter\def\csname PYG@tok@nf\endcsname{\def\PYG@tc##1{\textcolor[rgb]{0.02,0.16,0.49}{##1}}}
\expandafter\def\csname PYG@tok@cs\endcsname{\def\PYG@tc##1{\textcolor[rgb]{0.25,0.50,0.56}{##1}}\def\PYG@bc##1{\setlength{\fboxsep}{0pt}\colorbox[rgb]{1.00,0.94,0.94}{\strut ##1}}}
\expandafter\def\csname PYG@tok@gu\endcsname{\let\PYG@bf=\textbf\def\PYG@tc##1{\textcolor[rgb]{0.50,0.00,0.50}{##1}}}
\expandafter\def\csname PYG@tok@mb\endcsname{\def\PYG@tc##1{\textcolor[rgb]{0.13,0.50,0.31}{##1}}}
\expandafter\def\csname PYG@tok@gh\endcsname{\let\PYG@bf=\textbf\def\PYG@tc##1{\textcolor[rgb]{0.00,0.00,0.50}{##1}}}
\expandafter\def\csname PYG@tok@gd\endcsname{\def\PYG@tc##1{\textcolor[rgb]{0.63,0.00,0.00}{##1}}}
\expandafter\def\csname PYG@tok@kt\endcsname{\def\PYG@tc##1{\textcolor[rgb]{0.56,0.13,0.00}{##1}}}
\expandafter\def\csname PYG@tok@ss\endcsname{\def\PYG@tc##1{\textcolor[rgb]{0.32,0.47,0.09}{##1}}}
\expandafter\def\csname PYG@tok@nl\endcsname{\let\PYG@bf=\textbf\def\PYG@tc##1{\textcolor[rgb]{0.00,0.13,0.44}{##1}}}
\expandafter\def\csname PYG@tok@kn\endcsname{\let\PYG@bf=\textbf\def\PYG@tc##1{\textcolor[rgb]{0.00,0.44,0.13}{##1}}}
\expandafter\def\csname PYG@tok@c\endcsname{\let\PYG@it=\textit\def\PYG@tc##1{\textcolor[rgb]{0.25,0.50,0.56}{##1}}}
\expandafter\def\csname PYG@tok@m\endcsname{\def\PYG@tc##1{\textcolor[rgb]{0.13,0.50,0.31}{##1}}}
\expandafter\def\csname PYG@tok@gs\endcsname{\let\PYG@bf=\textbf}
\expandafter\def\csname PYG@tok@gp\endcsname{\let\PYG@bf=\textbf\def\PYG@tc##1{\textcolor[rgb]{0.78,0.36,0.04}{##1}}}
\expandafter\def\csname PYG@tok@nd\endcsname{\let\PYG@bf=\textbf\def\PYG@tc##1{\textcolor[rgb]{0.33,0.33,0.33}{##1}}}
\expandafter\def\csname PYG@tok@sr\endcsname{\def\PYG@tc##1{\textcolor[rgb]{0.14,0.33,0.53}{##1}}}
\expandafter\def\csname PYG@tok@sd\endcsname{\let\PYG@it=\textit\def\PYG@tc##1{\textcolor[rgb]{0.25,0.44,0.63}{##1}}}
\expandafter\def\csname PYG@tok@kd\endcsname{\let\PYG@bf=\textbf\def\PYG@tc##1{\textcolor[rgb]{0.00,0.44,0.13}{##1}}}
\expandafter\def\csname PYG@tok@s\endcsname{\def\PYG@tc##1{\textcolor[rgb]{0.25,0.44,0.63}{##1}}}
\expandafter\def\csname PYG@tok@se\endcsname{\let\PYG@bf=\textbf\def\PYG@tc##1{\textcolor[rgb]{0.25,0.44,0.63}{##1}}}
\expandafter\def\csname PYG@tok@c1\endcsname{\let\PYG@it=\textit\def\PYG@tc##1{\textcolor[rgb]{0.25,0.50,0.56}{##1}}}
\expandafter\def\csname PYG@tok@nb\endcsname{\def\PYG@tc##1{\textcolor[rgb]{0.00,0.44,0.13}{##1}}}
\expandafter\def\csname PYG@tok@sb\endcsname{\def\PYG@tc##1{\textcolor[rgb]{0.25,0.44,0.63}{##1}}}
\expandafter\def\csname PYG@tok@mf\endcsname{\def\PYG@tc##1{\textcolor[rgb]{0.13,0.50,0.31}{##1}}}
\expandafter\def\csname PYG@tok@na\endcsname{\def\PYG@tc##1{\textcolor[rgb]{0.25,0.44,0.63}{##1}}}
\expandafter\def\csname PYG@tok@s2\endcsname{\def\PYG@tc##1{\textcolor[rgb]{0.25,0.44,0.63}{##1}}}
\expandafter\def\csname PYG@tok@err\endcsname{\def\PYG@bc##1{\setlength{\fboxsep}{0pt}\fcolorbox[rgb]{1.00,0.00,0.00}{1,1,1}{\strut ##1}}}
\expandafter\def\csname PYG@tok@vc\endcsname{\def\PYG@tc##1{\textcolor[rgb]{0.73,0.38,0.84}{##1}}}
\expandafter\def\csname PYG@tok@k\endcsname{\let\PYG@bf=\textbf\def\PYG@tc##1{\textcolor[rgb]{0.00,0.44,0.13}{##1}}}
\expandafter\def\csname PYG@tok@kr\endcsname{\let\PYG@bf=\textbf\def\PYG@tc##1{\textcolor[rgb]{0.00,0.44,0.13}{##1}}}
\expandafter\def\csname PYG@tok@bp\endcsname{\def\PYG@tc##1{\textcolor[rgb]{0.00,0.44,0.13}{##1}}}
\expandafter\def\csname PYG@tok@il\endcsname{\def\PYG@tc##1{\textcolor[rgb]{0.13,0.50,0.31}{##1}}}
\expandafter\def\csname PYG@tok@vg\endcsname{\def\PYG@tc##1{\textcolor[rgb]{0.73,0.38,0.84}{##1}}}
\expandafter\def\csname PYG@tok@ge\endcsname{\let\PYG@it=\textit}
\expandafter\def\csname PYG@tok@w\endcsname{\def\PYG@tc##1{\textcolor[rgb]{0.73,0.73,0.73}{##1}}}
\expandafter\def\csname PYG@tok@ne\endcsname{\def\PYG@tc##1{\textcolor[rgb]{0.00,0.44,0.13}{##1}}}
\expandafter\def\csname PYG@tok@sh\endcsname{\def\PYG@tc##1{\textcolor[rgb]{0.25,0.44,0.63}{##1}}}
\expandafter\def\csname PYG@tok@o\endcsname{\def\PYG@tc##1{\textcolor[rgb]{0.40,0.40,0.40}{##1}}}
\expandafter\def\csname PYG@tok@sc\endcsname{\def\PYG@tc##1{\textcolor[rgb]{0.25,0.44,0.63}{##1}}}
\expandafter\def\csname PYG@tok@gi\endcsname{\def\PYG@tc##1{\textcolor[rgb]{0.00,0.63,0.00}{##1}}}
\expandafter\def\csname PYG@tok@no\endcsname{\def\PYG@tc##1{\textcolor[rgb]{0.38,0.68,0.84}{##1}}}
\expandafter\def\csname PYG@tok@gt\endcsname{\def\PYG@tc##1{\textcolor[rgb]{0.00,0.27,0.87}{##1}}}
\expandafter\def\csname PYG@tok@nn\endcsname{\let\PYG@bf=\textbf\def\PYG@tc##1{\textcolor[rgb]{0.05,0.52,0.71}{##1}}}
\expandafter\def\csname PYG@tok@nv\endcsname{\def\PYG@tc##1{\textcolor[rgb]{0.73,0.38,0.84}{##1}}}
\expandafter\def\csname PYG@tok@cp\endcsname{\def\PYG@tc##1{\textcolor[rgb]{0.00,0.44,0.13}{##1}}}
\expandafter\def\csname PYG@tok@go\endcsname{\def\PYG@tc##1{\textcolor[rgb]{0.20,0.20,0.20}{##1}}}
\expandafter\def\csname PYG@tok@gr\endcsname{\def\PYG@tc##1{\textcolor[rgb]{1.00,0.00,0.00}{##1}}}
\expandafter\def\csname PYG@tok@kc\endcsname{\let\PYG@bf=\textbf\def\PYG@tc##1{\textcolor[rgb]{0.00,0.44,0.13}{##1}}}
\expandafter\def\csname PYG@tok@sx\endcsname{\def\PYG@tc##1{\textcolor[rgb]{0.78,0.36,0.04}{##1}}}
\expandafter\def\csname PYG@tok@nt\endcsname{\let\PYG@bf=\textbf\def\PYG@tc##1{\textcolor[rgb]{0.02,0.16,0.45}{##1}}}
\expandafter\def\csname PYG@tok@mi\endcsname{\def\PYG@tc##1{\textcolor[rgb]{0.13,0.50,0.31}{##1}}}
\expandafter\def\csname PYG@tok@cm\endcsname{\let\PYG@it=\textit\def\PYG@tc##1{\textcolor[rgb]{0.25,0.50,0.56}{##1}}}
\expandafter\def\csname PYG@tok@nc\endcsname{\let\PYG@bf=\textbf\def\PYG@tc##1{\textcolor[rgb]{0.05,0.52,0.71}{##1}}}
\expandafter\def\csname PYG@tok@vi\endcsname{\def\PYG@tc##1{\textcolor[rgb]{0.73,0.38,0.84}{##1}}}
\expandafter\def\csname PYG@tok@ow\endcsname{\let\PYG@bf=\textbf\def\PYG@tc##1{\textcolor[rgb]{0.00,0.44,0.13}{##1}}}
\expandafter\def\csname PYG@tok@s1\endcsname{\def\PYG@tc##1{\textcolor[rgb]{0.25,0.44,0.63}{##1}}}
\expandafter\def\csname PYG@tok@ni\endcsname{\let\PYG@bf=\textbf\def\PYG@tc##1{\textcolor[rgb]{0.84,0.33,0.22}{##1}}}

\def\PYGZbs{\char`\\}
\def\PYGZus{\char`\_}
\def\PYGZob{\char`\{}
\def\PYGZcb{\char`\}}
\def\PYGZca{\char`\^}
\def\PYGZam{\char`\&}
\def\PYGZlt{\char`\<}
\def\PYGZgt{\char`\>}
\def\PYGZsh{\char`\#}
\def\PYGZpc{\char`\%}
\def\PYGZdl{\char`\$}
\def\PYGZhy{\char`\-}
\def\PYGZsq{\char`\'}
\def\PYGZdq{\char`\"}
\def\PYGZti{\char`\~}
% for compatibility with earlier versions
\def\PYGZat{@}
\def\PYGZlb{[}
\def\PYGZrb{]}
\makeatother

\renewcommand\PYGZsq{\textquotesingle}

\begin{document}

\maketitle
\tableofcontents
\phantomsection\label{index::doc}


Documentation pour le cours ENR835
TESTS

Contents:


\chapter{includeme}
\label{includeme:enr835-technologies-des-systemes-solaires}\label{includeme:includeme}\label{includeme::doc}

\chapter{ENR835}
\label{ENR835::doc}\label{ENR835:enr835}
\href{https://ena.etsmtl.ca/course/view.php?id=3388/}{Site web du cours}

\href{http://gearju.com/225768904560/Data/Engineering/Solar/Solar\%20Engineering\%20of\%20Thermal\%20Processes,\%204th\%20Edition\%20-\%20GearTeam.pdf}{Livre de référence}


\section{Cours 1}
\label{ENR835:cours-1}
\textbf{Introduction aux systèmes solaires. Radiation solaire extra-atmosphère}
\begin{itemize}
\item {} 
Exemples

\item {} 
Notes

\end{itemize}


\section{Cours 2}
\label{ENR835:cours-2}
\textbf{Radiation solaire disponible sur plan horizontal.  Calcul des ombrages}


\section{Cours 3}
\label{ENR835:cours-3}
\textbf{Radiation sur plans inclinés:  modèles isotropes, HD,HDKR,Perez}


\section{Cours 4}
\label{ENR835:cours-4}
\textbf{Introduction sur TRNSYS}


\section{Cours 5}
\label{ENR835:cours-5}
\textbf{Théorie des capteurs solaires plats}

Exemple 3.2 à été terminé au début du cours.


\section{Cours 6}
\label{ENR835:cours-6}
\textbf{Théorie des capteurs solaires plats-calculs des pertes}


\section{Cours 7}
\label{ENR835:cours-7}
\textbf{Théorie des capteurs solaires plats - Capteurs sous-vide}


\section{Cours 8}
\label{ENR835:cours-8}
\textbf{Systèmes pressurisés ou à gravité-Stratégies de stockage}


\section{Cours 9}
\label{ENR835:cours-9}
\textbf{Conception des systèmes de chauffage (méthode f-chart)}


\section{Cours 10}
\label{ENR835:cours-10}
\textbf{Conception des systèmes de chauffage (méthode  du potentiel d’utilisation)}


\section{Cours 11}
\label{ENR835:cours-11}
\textbf{Analyse économique}


\section{Cours 12}
\label{ENR835:cours-12}
\textbf{Introduction à  la réfrigération solaire}


\section{Cours 13}
\label{ENR835:cours-13}
\textbf{Panneaux Photo-voltaïques}

{\hfill\includegraphics{python-powered.png}}


\chapter{Python}
\label{python::doc}\label{python:python}
Les programmes pythons sont documenté sur cette page. blablabla
\phantomsection\label{python:module-solar_mod}\index{solar\_mod (module)}

\section{solar\_mod}
\label{python:solar-mod}
Pour mettre a jour le programme solar\_mod:

\begin{Verbatim}[commandchars=\\\{\}]
cd C:\PYGZbs{}Users\PYGZbs{}Antoine\PYGZbs{}Documents\PYGZbs{}GitHub\PYGZbs{}Solar\PYGZus{}mod
python setup.py install
\end{Verbatim}


\subsection{Angle d'azimuth solaire}
\label{python:angle-d-azimuth-solaire}\index{azimuth\_solaire() (in module solar\_mod)}

\begin{fulllineitems}
\phantomsection\label{python:solar_mod.azimuth_solaire}\pysiglinewithargsret{\code{solar\_mod.}\bfcode{azimuth\_solaire}}{\emph{thes}, \emph{delta}, \emph{phi}, \emph{ome}}{}
r'`' Petite descrition de la fonction icluant formules...
\begin{quote}\begin{description}
\item[{Parameters}] \leavevmode
\textbf{\texttt{param}} (\href{https://docs.python.org/library/functions.html\#float}{\emph{float}}) -- Ce paramètre est...

\item[{Returns}] \leavevmode
\textbf{param} -- Ce paramètre est...

\item[{Return type}] \leavevmode
\href{https://docs.python.org/library/functions.html\#float}{float}

\end{description}\end{quote}
\paragraph{Notes}

Notes s'il en as.
\paragraph{References}

La référence.

\end{fulllineitems}



\subsection{Angle de déclinaison solaire (delta)}
\label{python:angle-de-declinaison-solaire-delta}\index{decl\_solaire() (in module solar\_mod)}

\begin{fulllineitems}
\phantomsection\label{python:solar_mod.decl_solaire}\pysiglinewithargsret{\code{solar\_mod.}\bfcode{decl\_solaire}}{\emph{n}, \emph{cas=1}}{}
r'`' Petite descrition de la fonction icluant formules...
\begin{quote}\begin{description}
\item[{Parameters}] \leavevmode
\textbf{\texttt{param}} (\href{https://docs.python.org/library/functions.html\#float}{\emph{float}}) -- Ce paramètre est...

\item[{Returns}] \leavevmode
\textbf{param} -- Ce paramètre est...

\item[{Return type}] \leavevmode
\href{https://docs.python.org/library/functions.html\#float}{float}

\end{description}\end{quote}
\paragraph{Notes}

Notes s'il en as.
\paragraph{References}

La référence.

\end{fulllineitems}



\subsection{Angle horaire (omega)}
\label{python:angle-horaire-omega}\index{angle\_horaire() (in module solar\_mod)}

\begin{fulllineitems}
\phantomsection\label{python:solar_mod.angle_horaire}\pysiglinewithargsret{\code{solar\_mod.}\bfcode{angle\_horaire}}{\emph{sol\_t}}{}
Angle horaire en fonction du temps solaire en heures minutes
\begin{quote}\begin{description}
\item[{Parameters}] \leavevmode
\textbf{\texttt{sol\_t}} (\href{https://docs.python.org/library/functions.html\#float}{\emph{float}}) -- Temps solaire en heures minutes de 0 à 24 hr

\item[{Returns}] \leavevmode
\textbf{ome} -- Angle horaire

\item[{Return type}] \leavevmode
\href{https://docs.python.org/library/functions.html\#float}{float}

\end{description}\end{quote}
\paragraph{Notes}

La terre tourne 15 deg par heure.
ome = 0 à midi solaire (\textgreater{}0 PM, \textless{}0 AM)
\paragraph{References}

Solar Engineering of Thermal Procecess (1.6.10)

\end{fulllineitems}



\subsection{Angle réfléchi}
\label{python:angle-reflechi}\index{angle\_reflechi() (in module solar\_mod)}

\begin{fulllineitems}
\phantomsection\label{python:solar_mod.angle_reflechi}\pysiglinewithargsret{\code{solar\_mod.}\bfcode{angle\_reflechi}}{\emph{beta}}{}
r'`' Petite descrition de la fonction icluant formules...
\begin{quote}\begin{description}
\item[{Parameters}] \leavevmode
\textbf{\texttt{param}} (\href{https://docs.python.org/library/functions.html\#float}{\emph{float}}) -- Ce paramètre est...

\item[{Returns}] \leavevmode
\textbf{param} -- Ce paramètre est...

\item[{Return type}] \leavevmode
\href{https://docs.python.org/library/functions.html\#float}{float}

\end{description}\end{quote}
\paragraph{Notes}

Notes s'il en as.
\paragraph{References}

La référence.

\end{fulllineitems}



\subsection{Angle d'altitude solaire (omega s)}
\label{python:angle-d-altitude-solaire-omega-s}\index{angle\_sunset() (in module solar\_mod)}

\begin{fulllineitems}
\phantomsection\label{python:solar_mod.angle_sunset}\pysiglinewithargsret{\code{solar\_mod.}\bfcode{angle\_sunset}}{\emph{phi}, \emph{delta}}{}
r'`' Petite descrition de la fonction icluant formules...
\begin{quote}\begin{description}
\item[{Parameters}] \leavevmode
\textbf{\texttt{param}} (\href{https://docs.python.org/library/functions.html\#float}{\emph{float}}) -- Ce paramètre est...

\item[{Returns}] \leavevmode
\textbf{param} -- Ce paramètre est...

\item[{Return type}] \leavevmode
\href{https://docs.python.org/library/functions.html\#float}{float}

\end{description}\end{quote}
\paragraph{Notes}

Notes s'il en as.
\paragraph{References}

La référence.

\end{fulllineitems}



\subsection{Normale solaire (theta rad)}
\label{python:normale-solaire-theta-rad}\index{normale\_solaire() (in module solar\_mod)}

\begin{fulllineitems}
\phantomsection\label{python:solar_mod.normale_solaire}\pysiglinewithargsret{\code{solar\_mod.}\bfcode{normale\_solaire}}{\emph{delt}, \emph{phi}, \emph{ome}, \emph{beta}, \emph{gam}}{}
r'`' Petite descrition de la fonction icluant formules...
\begin{quote}\begin{description}
\item[{Parameters}] \leavevmode
\textbf{\texttt{param}} (\href{https://docs.python.org/library/functions.html\#float}{\emph{float}}) -- Ce paramètre est...

\item[{Returns}] \leavevmode
\textbf{param} -- Ce paramètre est...

\item[{Return type}] \leavevmode
\href{https://docs.python.org/library/functions.html\#float}{float}

\end{description}\end{quote}
\paragraph{Notes}

Notes s'il en as.
\paragraph{References}

La référence.

\end{fulllineitems}



\subsection{Angle...}
\label{python:angle}\index{alp\_alpn() (in module solar\_mod)}

\begin{fulllineitems}
\phantomsection\label{python:solar_mod.alp_alpn}\pysiglinewithargsret{\code{solar\_mod.}\bfcode{alp\_alpn}}{\emph{the}}{}
r'`' Petite descrition de la fonction icluant formules...
\begin{quote}\begin{description}
\item[{Parameters}] \leavevmode
\textbf{\texttt{param}} (\href{https://docs.python.org/library/functions.html\#float}{\emph{float}}) -- Ce paramètre est...

\item[{Returns}] \leavevmode
\textbf{param} -- Ce paramètre est...

\item[{Return type}] \leavevmode
\href{https://docs.python.org/library/functions.html\#float}{float}

\end{description}\end{quote}
\paragraph{Notes}

Notes s'il en as.
\paragraph{References}

La référence.

\end{fulllineitems}



\subsection{Angle diffusion}
\label{python:angle-diffusion}\index{angle\_diffus() (in module solar\_mod)}

\begin{fulllineitems}
\phantomsection\label{python:solar_mod.angle_diffus}\pysiglinewithargsret{\code{solar\_mod.}\bfcode{angle\_diffus}}{\emph{beta}}{}
r'`' Petite descrition de la fonction icluant formules...
\begin{quote}\begin{description}
\item[{Parameters}] \leavevmode
\textbf{\texttt{param}} (\href{https://docs.python.org/library/functions.html\#float}{\emph{float}}) -- Ce paramètre est...

\item[{Returns}] \leavevmode
\textbf{param} -- Ce paramètre est...

\item[{Return type}] \leavevmode
\href{https://docs.python.org/library/functions.html\#float}{float}

\end{description}\end{quote}
\paragraph{Notes}

Notes s'il en as.
\paragraph{References}

La référence.

\end{fulllineitems}

\index{Calcul\_Ka() (in module solar\_mod)}

\begin{fulllineitems}
\phantomsection\label{python:solar_mod.Calcul_Ka}\pysiglinewithargsret{\code{solar\_mod.}\bfcode{Calcul\_Ka}}{\emph{It}, \emph{Itb}, \emph{Itd}, \emph{Itr}, \emph{theb}, \emph{bo}, \emph{beta}}{}
r'`' Petite descrition de la fonction icluant formules...
\begin{quote}\begin{description}
\item[{Parameters}] \leavevmode
\textbf{\texttt{param}} (\href{https://docs.python.org/library/functions.html\#float}{\emph{float}}) -- Ce paramètre est...

\item[{Returns}] \leavevmode
\textbf{param} -- Ce paramètre est...

\item[{Return type}] \leavevmode
\href{https://docs.python.org/library/functions.html\#float}{float}

\end{description}\end{quote}
\paragraph{Notes}

Notes s'il en as.
\paragraph{References}

La référence.

\end{fulllineitems}

\index{decl\_solaire() (in module solar\_mod)}

\begin{fulllineitems}
\pysiglinewithargsret{\code{solar\_mod.}\bfcode{decl\_solaire}}{\emph{n}, \emph{cas=1}}{}
r'`' Petite descrition de la fonction icluant formules...
\begin{quote}\begin{description}
\item[{Parameters}] \leavevmode
\textbf{\texttt{param}} (\href{https://docs.python.org/library/functions.html\#float}{\emph{float}}) -- Ce paramètre est...

\item[{Returns}] \leavevmode
\textbf{param} -- Ce paramètre est...

\item[{Return type}] \leavevmode
\href{https://docs.python.org/library/functions.html\#float}{float}

\end{description}\end{quote}
\paragraph{Notes}

Notes s'il en as.
\paragraph{References}

La référence.

\end{fulllineitems}

\index{normale\_solaire2() (in module solar\_mod)}

\begin{fulllineitems}
\phantomsection\label{python:solar_mod.normale_solaire2}\pysiglinewithargsret{\code{solar\_mod.}\bfcode{normale\_solaire2}}{\emph{thez}, \emph{gams}, \emph{beta}, \emph{gam}}{}
r'`' Petite descrition de la fonction icluant formules...
\begin{quote}\begin{description}
\item[{Parameters}] \leavevmode
\textbf{\texttt{param}} (\href{https://docs.python.org/library/functions.html\#float}{\emph{float}}) -- Ce paramètre est...

\item[{Returns}] \leavevmode
\textbf{param} -- Ce paramètre est...

\item[{Return type}] \leavevmode
\href{https://docs.python.org/library/functions.html\#float}{float}

\end{description}\end{quote}
\paragraph{Notes}

Notes s'il en as.
\paragraph{References}

La référence.

\end{fulllineitems}

\index{zenith\_solaire() (in module solar\_mod)}

\begin{fulllineitems}
\phantomsection\label{python:solar_mod.zenith_solaire}\pysiglinewithargsret{\code{solar\_mod.}\bfcode{zenith\_solaire}}{\emph{phi}, \emph{delt}, \emph{ome}}{}
r'`' Petite descrition de la fonction icluant formules...
\begin{quote}\begin{description}
\item[{Parameters}] \leavevmode
\textbf{\texttt{param}} (\href{https://docs.python.org/library/functions.html\#float}{\emph{float}}) -- Ce paramètre est...

\item[{Returns}] \leavevmode
\textbf{param} -- Ce paramètre est...

\item[{Return type}] \leavevmode
\href{https://docs.python.org/library/functions.html\#float}{float}

\end{description}\end{quote}
\paragraph{Notes}

Notes s'il en as.
\paragraph{References}

La référence.

\end{fulllineitems}

\index{heure\_angle() (in module solar\_mod)}

\begin{fulllineitems}
\phantomsection\label{python:solar_mod.heure_angle}\pysiglinewithargsret{\code{solar\_mod.}\bfcode{heure\_angle}}{\emph{ome}}{}
r'`' Petite descrition de la fonction icluant formules...
\begin{quote}\begin{description}
\item[{Parameters}] \leavevmode
\textbf{\texttt{param}} (\href{https://docs.python.org/library/functions.html\#float}{\emph{float}}) -- Ce paramètre est...

\item[{Returns}] \leavevmode
\textbf{param} -- Ce paramètre est...

\item[{Return type}] \leavevmode
\href{https://docs.python.org/library/functions.html\#float}{float}

\end{description}\end{quote}
\paragraph{Notes}

Notes s'il en as.
\paragraph{References}

La référence.

\end{fulllineitems}

\index{heure\_solaire() (in module solar\_mod)}

\begin{fulllineitems}
\phantomsection\label{python:solar_mod.heure_solaire}\pysiglinewithargsret{\code{solar\_mod.}\bfcode{heure\_solaire}}{\emph{lon}, \emph{Lst}, \emph{del\_h}, \emph{st}}{}
Heure solaire (sol\_t)
\begin{quote}\begin{description}
\item[{Parameters}] \leavevmode\begin{itemize}
\item {} 
\textbf{\texttt{lon}} (\href{https://docs.python.org/library/functions.html\#float}{\emph{float}}) -- longitude (lon.deg, lon.min )

\item {} 
\textbf{\texttt{Lst}} (\href{https://docs.python.org/library/functions.html\#float}{\emph{float}}) -- longitude du méridien de l'heure (-180 à 180, ex: -75 eastern time)

\item {} 
\textbf{\texttt{del\_h}} (\href{https://docs.python.org/library/functions.html\#float}{\emph{float}}) -- difference entre l'heure legale et l'heure strandard

\item {} 
\textbf{\texttt{st}} (\href{https://docs.python.org/library/functions.html\#float}{\emph{float}}) -- \begin{description}
\item[{temps legal}] \leavevmode
st\_jour : jour de l'année
st\_hr : heure , st\_min : minute

\end{description}


\end{itemize}

\item[{Returns}] \leavevmode
\textbf{param} -- Ce paramètre est...

\item[{Return type}] \leavevmode
\href{https://docs.python.org/library/functions.html\#float}{float}

\end{description}\end{quote}
\paragraph{Notes}

del\_h : Amerique (0 hiver , 1 ete),Europe (1 hiver , 2 ete)
\paragraph{References}

Solar Engineering of Thermal Procecess

\end{fulllineitems}

\index{heure\_legale() (in module solar\_mod)}

\begin{fulllineitems}
\phantomsection\label{python:solar_mod.heure_legale}\pysiglinewithargsret{\code{solar\_mod.}\bfcode{heure\_legale}}{\emph{lon}, \emph{Lst}, \emph{del\_h}, \emph{sol\_t}}{}
r'`' Petite descrition de la fonction icluant formules...
\begin{quote}\begin{description}
\item[{Parameters}] \leavevmode
\textbf{\texttt{param}} (\href{https://docs.python.org/library/functions.html\#float}{\emph{float}}) -- Ce paramètre est...

\item[{Returns}] \leavevmode
\textbf{param} -- Ce paramètre est...

\item[{Return type}] \leavevmode
\href{https://docs.python.org/library/functions.html\#float}{float}

\end{description}\end{quote}
\paragraph{Notes}

Notes s'il en as.
\paragraph{References}

La référence.

\end{fulllineitems}

\index{fct\_hl() (in module solar\_mod)}

\begin{fulllineitems}
\phantomsection\label{python:solar_mod.fct_hl}\pysiglinewithargsret{\code{solar\_mod.}\bfcode{fct\_hl}}{\emph{x}, \emph{n}, \emph{lon}, \emph{Lst}, \emph{del\_h}, \emph{tt}}{}
r'`' Petite descrition de la fonction icluant formules...
\begin{quote}\begin{description}
\item[{Parameters}] \leavevmode
\textbf{\texttt{param}} (\href{https://docs.python.org/library/functions.html\#float}{\emph{float}}) -- Ce paramètre est...

\item[{Returns}] \leavevmode
\textbf{param} -- Ce paramètre est...

\item[{Return type}] \leavevmode
\href{https://docs.python.org/library/functions.html\#float}{float}

\end{description}\end{quote}
\paragraph{Notes}

Notes s'il en as.
\paragraph{References}

La référence.

\end{fulllineitems}

\index{fchart() (in module solar\_mod)}

\begin{fulllineitems}
\phantomsection\label{python:solar_mod.fchart}\pysiglinewithargsret{\code{solar\_mod.}\bfcode{fchart}}{\emph{X}, \emph{Y}}{}
r'`' Petite descrition de la fonction icluant formules...
\begin{quote}\begin{description}
\item[{Parameters}] \leavevmode
\textbf{\texttt{param}} (\href{https://docs.python.org/library/functions.html\#float}{\emph{float}}) -- Ce paramètre est...

\item[{Returns}] \leavevmode
\textbf{param} -- Ce paramètre est...

\item[{Return type}] \leavevmode
\href{https://docs.python.org/library/functions.html\#float}{float}

\end{description}\end{quote}
\paragraph{Notes}

Notes s'il en as.
\paragraph{References}

La référence.

\end{fulllineitems}

\index{fchart\_air() (in module solar\_mod)}

\begin{fulllineitems}
\phantomsection\label{python:solar_mod.fchart_air}\pysiglinewithargsret{\code{solar\_mod.}\bfcode{fchart\_air}}{\emph{X}, \emph{Y}}{}
r'`' Petite descrition de la fonction icluant formules...
\begin{quote}\begin{description}
\item[{Parameters}] \leavevmode
\textbf{\texttt{param}} (\href{https://docs.python.org/library/functions.html\#float}{\emph{float}}) -- Ce paramètre est...

\item[{Returns}] \leavevmode
\textbf{param} -- Ce paramètre est...

\item[{Return type}] \leavevmode
\href{https://docs.python.org/library/functions.html\#float}{float}

\end{description}\end{quote}
\paragraph{Notes}

Notes s'il en as.
\paragraph{References}

La référence.

\end{fulllineitems}

\index{duree\_jour() (in module solar\_mod)}

\begin{fulllineitems}
\phantomsection\label{python:solar_mod.duree_jour}\pysiglinewithargsret{\code{solar\_mod.}\bfcode{duree\_jour}}{\emph{n}, \emph{phi}}{}
r'`' Petite descrition de la fonction icluant formules...
\begin{quote}\begin{description}
\item[{Parameters}] \leavevmode
\textbf{\texttt{param}} (\href{https://docs.python.org/library/functions.html\#float}{\emph{float}}) -- Ce paramètre est...

\item[{Returns}] \leavevmode
\textbf{param} -- Ce paramètre est...

\item[{Return type}] \leavevmode
\href{https://docs.python.org/library/functions.html\#float}{float}

\end{description}\end{quote}
\paragraph{Notes}

Notes s'il en as.
\paragraph{References}

La référence.

\end{fulllineitems}

\index{duree\_jour\_mod() (in module solar\_mod)}

\begin{fulllineitems}
\phantomsection\label{python:solar_mod.duree_jour_mod}\pysiglinewithargsret{\code{solar\_mod.}\bfcode{duree\_jour\_mod}}{\emph{n}, \emph{phi}}{}
r'`' Petite descrition de la fonction icluant formules...
\begin{quote}\begin{description}
\item[{Parameters}] \leavevmode
\textbf{\texttt{param}} (\href{https://docs.python.org/library/functions.html\#float}{\emph{float}}) -- Ce paramètre est...

\item[{Returns}] \leavevmode
\textbf{param} -- Ce paramètre est...

\item[{Return type}] \leavevmode
\href{https://docs.python.org/library/functions.html\#float}{float}

\end{description}\end{quote}
\paragraph{Notes}

Notes s'il en as.
\paragraph{References}

La référence.

\end{fulllineitems}

\index{irradiation\_extraterrestre\_normale() (in module solar\_mod)}

\begin{fulllineitems}
\phantomsection\label{python:solar_mod.irradiation_extraterrestre_normale}\pysiglinewithargsret{\code{solar\_mod.}\bfcode{irradiation\_extraterrestre\_normale}}{\emph{n}}{}
r'`' Petite descrition de la fonction icluant formules...
\begin{quote}\begin{description}
\item[{Parameters}] \leavevmode
\textbf{\texttt{param}} (\href{https://docs.python.org/library/functions.html\#float}{\emph{float}}) -- Ce paramètre est...

\item[{Returns}] \leavevmode
\textbf{param} -- Ce paramètre est...

\item[{Return type}] \leavevmode
\href{https://docs.python.org/library/functions.html\#float}{float}

\end{description}\end{quote}
\paragraph{Notes}

Notes s'il en as.
\paragraph{References}

La référence.

\end{fulllineitems}

\index{irradiation\_extraterrestre() (in module solar\_mod)}

\begin{fulllineitems}
\phantomsection\label{python:solar_mod.irradiation_extraterrestre}\pysiglinewithargsret{\code{solar\_mod.}\bfcode{irradiation\_extraterrestre}}{\emph{n}, \emph{thez}}{}
Irradiation extraterrestre incidente sur un plan normal au jour n de l'année
\begin{quote}\begin{description}
\item[{Parameters}] \leavevmode\begin{itemize}
\item {} 
\textbf{\texttt{n}} (\href{https://docs.python.org/library/functions.html\#float}{\emph{float}}) -- Jour de l'année

\item {} 
\textbf{\texttt{thez}} (\href{https://docs.python.org/library/functions.html\#float}{\emph{float}}) -- Angle du capteur

\end{itemize}

\item[{Returns}] \leavevmode
\textbf{G} -- Irradiation extraterrestre

\item[{Return type}] \leavevmode
\href{https://docs.python.org/library/functions.html\#float}{float}

\end{description}\end{quote}
\paragraph{References}

Solar Engineering of Thermal Procecess (1.4.1a)

\end{fulllineitems}

\index{irradiation\_extraterrestre\_jour() (in module solar\_mod)}

\begin{fulllineitems}
\phantomsection\label{python:solar_mod.irradiation_extraterrestre_jour}\pysiglinewithargsret{\code{solar\_mod.}\bfcode{irradiation\_extraterrestre\_jour}}{\emph{n}, \emph{phi}}{}
r'`' Petite descrition de la fonction icluant formules...
\begin{quote}\begin{description}
\item[{Parameters}] \leavevmode
\textbf{\texttt{param}} (\href{https://docs.python.org/library/functions.html\#float}{\emph{float}}) -- Ce paramètre est...

\item[{Returns}] \leavevmode
\textbf{param} -- Ce paramètre est...

\item[{Return type}] \leavevmode
\href{https://docs.python.org/library/functions.html\#float}{float}

\end{description}\end{quote}
\paragraph{Notes}

Notes s'il en as.
\paragraph{References}

La référence.

\end{fulllineitems}

\index{irradiation\_extraterrestre\_horaire() (in module solar\_mod)}

\begin{fulllineitems}
\phantomsection\label{python:solar_mod.irradiation_extraterrestre_horaire}\pysiglinewithargsret{\code{solar\_mod.}\bfcode{irradiation\_extraterrestre\_horaire}}{\emph{n}, \emph{phi}, \emph{ome1}, \emph{ome2}}{}
Irradiation solaire extraterrestre par heure.
\begin{quote}\begin{description}
\item[{Parameters}] \leavevmode\begin{itemize}
\item {} 
\textbf{\texttt{n}} (\href{https://docs.python.org/library/functions.html\#float}{\emph{float}}) -- Jour de lannee

\item {} 
\textbf{\texttt{phi}} (\href{https://docs.python.org/library/functions.html\#float}{\emph{float}}) -- Angle lattitude

\item {} 
\textbf{\texttt{ome1}} (\href{https://docs.python.org/library/functions.html\#float}{\emph{float}}) -- Angle horaire 1

\item {} 
\textbf{\texttt{ome2}} (\href{https://docs.python.org/library/functions.html\#float}{\emph{float}}) -- Angle horaire 2

\end{itemize}

\item[{Returns}] \leavevmode
\textbf{Io} -- L'irradiation solaire extraterrestre par heure

\item[{Return type}] \leavevmode
\href{https://docs.python.org/library/functions.html\#float}{float}

\end{description}\end{quote}
\paragraph{References}

Solar Engineering of Thermal Procecess (1.10.4)

\end{fulllineitems}

\index{irradiation\_extraterrestre\_jour\_moyen() (in module solar\_mod)}

\begin{fulllineitems}
\phantomsection\label{python:solar_mod.irradiation_extraterrestre_jour_moyen}\pysiglinewithargsret{\code{solar\_mod.}\bfcode{irradiation\_extraterrestre\_jour\_moyen}}{\emph{nmois}, \emph{phi}}{}
r'`' Petite descrition de la fonction icluant formules...
\begin{quote}\begin{description}
\item[{Parameters}] \leavevmode
\textbf{\texttt{param}} (\href{https://docs.python.org/library/functions.html\#float}{\emph{float}}) -- Ce paramètre est...

\item[{Returns}] \leavevmode
\textbf{param} -- Ce paramètre est...

\item[{Return type}] \leavevmode
\href{https://docs.python.org/library/functions.html\#float}{float}

\end{description}\end{quote}
\paragraph{Notes}

Notes s'il en as.
\paragraph{References}

La référence.

\end{fulllineitems}

\index{jour\_mois\_jour\_annee() (in module solar\_mod)}

\begin{fulllineitems}
\phantomsection\label{python:solar_mod.jour_mois_jour_annee}\pysiglinewithargsret{\code{solar\_mod.}\bfcode{jour\_mois\_jour\_annee}}{\emph{jour}, \emph{mois}}{}
fonction jour\_mois\_jour\_annee sert à convertir un jour du calendrier en jour annuel 1 à 365.
\begin{quote}\begin{description}
\item[{Parameters}] \leavevmode\begin{itemize}
\item {} 
\textbf{\texttt{jour}} (\href{https://docs.python.org/library/functions.html\#float}{\emph{float}}) -- Jour du mois

\item {} 
\textbf{\texttt{mois}} (\href{https://docs.python.org/library/functions.html\#float}{\emph{float}}) -- Mois de l'année

\end{itemize}

\item[{Returns}] \leavevmode
\textbf{mois} -- Mois de l'année

\item[{Return type}] \leavevmode
\href{https://docs.python.org/library/functions.html\#float}{float}

\end{description}\end{quote}
\paragraph{Notes}

fonction qui transforme une date en jour et mois
en jour de 1 à 365
Les mois doivent s'écrire
`jan';'fev';'mars';'avr';'mai';'juin';'juil';'aout';'sept';'oct
`;'nov';'dec'

\end{fulllineitems}

\index{jour\_annee\_jour\_mois() (in module solar\_mod)}

\begin{fulllineitems}
\phantomsection\label{python:solar_mod.jour_annee_jour_mois}\pysiglinewithargsret{\code{solar\_mod.}\bfcode{jour\_annee\_jour\_mois}}{\emph{n}}{}
r'`' Petite descrition de la fonction icluant formules...
\begin{quote}\begin{description}
\item[{Parameters}] \leavevmode
\textbf{\texttt{param}} (\href{https://docs.python.org/library/functions.html\#float}{\emph{float}}) -- Ce paramètre est...

\item[{Returns}] \leavevmode
\textbf{param} -- Ce paramètre est...

\item[{Return type}] \leavevmode
\href{https://docs.python.org/library/functions.html\#float}{float}

\end{description}\end{quote}
\paragraph{Notes}

Notes s'il en as.
\paragraph{References}

La référence.

\end{fulllineitems}

\index{Erbs\_horaire() (in module solar\_mod)}

\begin{fulllineitems}
\phantomsection\label{python:solar_mod.Erbs_horaire}\pysiglinewithargsret{\code{solar\_mod.}\bfcode{Erbs\_horaire}}{\emph{kt}}{}
r'`' Petite descrition de la fonction icluant formules...
\begin{quote}\begin{description}
\item[{Parameters}] \leavevmode
\textbf{\texttt{param}} (\href{https://docs.python.org/library/functions.html\#float}{\emph{float}}) -- Ce paramètre est...

\item[{Returns}] \leavevmode
\textbf{param} -- Ce paramètre est...

\item[{Return type}] \leavevmode
\href{https://docs.python.org/library/functions.html\#float}{float}

\end{description}\end{quote}
\paragraph{Notes}

Notes s'il en as.
\paragraph{References}

La référence.

\end{fulllineitems}

\index{Erbs\_jour() (in module solar\_mod)}

\begin{fulllineitems}
\phantomsection\label{python:solar_mod.Erbs_jour}\pysiglinewithargsret{\code{solar\_mod.}\bfcode{Erbs\_jour}}{\emph{kt}, \emph{ws}}{}
r'`' Petite descrition de la fonction icluant formules...
\begin{quote}\begin{description}
\item[{Parameters}] \leavevmode
\textbf{\texttt{param}} (\href{https://docs.python.org/library/functions.html\#float}{\emph{float}}) -- Ce paramètre est...

\item[{Returns}] \leavevmode
\textbf{param} -- Ce paramètre est...

\item[{Return type}] \leavevmode
\href{https://docs.python.org/library/functions.html\#float}{float}

\end{description}\end{quote}
\paragraph{Notes}

Notes s'il en as.
\paragraph{References}

La référence.

\end{fulllineitems}

\index{Erbs\_mois() (in module solar\_mod)}

\begin{fulllineitems}
\phantomsection\label{python:solar_mod.Erbs_mois}\pysiglinewithargsret{\code{solar\_mod.}\bfcode{Erbs\_mois}}{\emph{kt}, \emph{ws}}{}
r'`' Petite descrition de la fonction icluant formules...
\begin{quote}\begin{description}
\item[{Parameters}] \leavevmode
\textbf{\texttt{param}} (\href{https://docs.python.org/library/functions.html\#float}{\emph{float}}) -- Ce paramètre est...

\item[{Returns}] \leavevmode
\textbf{param} -- Ce paramètre est...

\item[{Return type}] \leavevmode
\href{https://docs.python.org/library/functions.html\#float}{float}

\end{description}\end{quote}
\paragraph{Notes}

Notes s'il en as.
\paragraph{References}

La référence.

\end{fulllineitems}

\index{Collares\_total() (in module solar\_mod)}

\begin{fulllineitems}
\phantomsection\label{python:solar_mod.Collares_total}\pysiglinewithargsret{\code{solar\_mod.}\bfcode{Collares\_total}}{\emph{ome}, \emph{omes}}{}
r'`' Petite descrition de la fonction icluant formules...
\begin{quote}\begin{description}
\item[{Parameters}] \leavevmode
\textbf{\texttt{param}} (\href{https://docs.python.org/library/functions.html\#float}{\emph{float}}) -- Ce paramètre est...

\item[{Returns}] \leavevmode
\textbf{param} -- Ce paramètre est...

\item[{Return type}] \leavevmode
\href{https://docs.python.org/library/functions.html\#float}{float}

\end{description}\end{quote}
\paragraph{Notes}

Notes s'il en as.
\paragraph{References}

La référence.

\end{fulllineitems}

\index{Collares\_diffus() (in module solar\_mod)}

\begin{fulllineitems}
\phantomsection\label{python:solar_mod.Collares_diffus}\pysiglinewithargsret{\code{solar\_mod.}\bfcode{Collares\_diffus}}{\emph{ome}, \emph{omes}}{}
r'`' Petite descrition de la fonction icluant formules...
\begin{quote}\begin{description}
\item[{Parameters}] \leavevmode
\textbf{\texttt{param}} (\href{https://docs.python.org/library/functions.html\#float}{\emph{float}}) -- Ce paramètre est...

\item[{Returns}] \leavevmode
\textbf{param} -- Ce paramètre est...

\item[{Return type}] \leavevmode
\href{https://docs.python.org/library/functions.html\#float}{float}

\end{description}\end{quote}
\paragraph{Notes}

Notes s'il en as.
\paragraph{References}

La référence.

\end{fulllineitems}

\index{Calcul\_pertes() (in module solar\_mod)}

\begin{fulllineitems}
\phantomsection\label{python:solar_mod.Calcul_pertes}\pysiglinewithargsret{\code{solar\_mod.}\bfcode{Calcul\_pertes}}{\emph{T1c}, \emph{beta}, \emph{H}, \emph{Y}, \emph{uinf}, \emph{Tinfc}, \emph{Tskyc}, \emph{Lair}, \emph{e1}, \emph{e2}, \emph{e3=-1}}{}
r'`' Petite descrition de la fonction icluant formules...
\begin{quote}\begin{description}
\item[{Parameters}] \leavevmode
\textbf{\texttt{param}} (\href{https://docs.python.org/library/functions.html\#float}{\emph{float}}) -- Ce paramètre est...

\item[{Returns}] \leavevmode
\textbf{param} -- Ce paramètre est...

\item[{Return type}] \leavevmode
\href{https://docs.python.org/library/functions.html\#float}{float}

\end{description}\end{quote}
\paragraph{Notes}

Notes s'il en as.
\paragraph{References}

La référence.

\end{fulllineitems}

\index{U\_Klein() (in module solar\_mod)}

\begin{fulllineitems}
\phantomsection\label{python:solar_mod.U_Klein}\pysiglinewithargsret{\code{solar\_mod.}\bfcode{U\_Klein}}{\emph{T\_pc}, \emph{T\_ac}, \emph{Slope}, \emph{h}, \emph{Emitt}, \emph{emig}, \emph{n}}{}
r'`' Petite descrition de la fonction icluant formules...
\begin{quote}\begin{description}
\item[{Parameters}] \leavevmode
\textbf{\texttt{param}} (\href{https://docs.python.org/library/functions.html\#float}{\emph{float}}) -- Ce paramètre est...

\item[{Returns}] \leavevmode
\textbf{param} -- Ce paramètre est...

\item[{Return type}] \leavevmode
\href{https://docs.python.org/library/functions.html\#float}{float}

\end{description}\end{quote}
\paragraph{Notes}

Notes s'il en as.
\paragraph{References}

La référence.

\end{fulllineitems}

\index{calcul\_Rb() (in module solar\_mod)}

\begin{fulllineitems}
\phantomsection\label{python:solar_mod.calcul_Rb}\pysiglinewithargsret{\code{solar\_mod.}\bfcode{calcul\_Rb}}{\emph{phi}, \emph{n}, \emph{ome}, \emph{beta}, \emph{gam}}{}
r'`' Petite descrition de la fonction icluant formules...
\begin{quote}\begin{description}
\item[{Parameters}] \leavevmode
\textbf{\texttt{param}} (\href{https://docs.python.org/library/functions.html\#float}{\emph{float}}) -- Ce paramètre est...

\item[{Returns}] \leavevmode
\textbf{param} -- Ce paramètre est...

\item[{Return type}] \leavevmode
\href{https://docs.python.org/library/functions.html\#float}{float}

\end{description}\end{quote}
\paragraph{Notes}

Notes s'il en as.
\paragraph{References}

La référence.

\end{fulllineitems}

\index{calcul\_Rb\_mois() (in module solar\_mod)}

\begin{fulllineitems}
\phantomsection\label{python:solar_mod.calcul_Rb_mois}\pysiglinewithargsret{\code{solar\_mod.}\bfcode{calcul\_Rb\_mois}}{\emph{phi}, \emph{n}, \emph{beta}, \emph{gam}}{}
r'`' Petite descrition de la fonction icluant formules...
\begin{quote}\begin{description}
\item[{Parameters}] \leavevmode
\textbf{\texttt{param}} (\href{https://docs.python.org/library/functions.html\#float}{\emph{float}}) -- Ce paramètre est...

\item[{Returns}] \leavevmode
\textbf{param} -- Ce paramètre est...

\item[{Return type}] \leavevmode
\href{https://docs.python.org/library/functions.html\#float}{float}

\end{description}\end{quote}
\paragraph{Notes}

Notes s'il en as.
\paragraph{References}

La référence.

\end{fulllineitems}

\index{modele\_isotropique() (in module solar\_mod)}

\begin{fulllineitems}
\phantomsection\label{python:solar_mod.modele_isotropique}\pysiglinewithargsret{\code{solar\_mod.}\bfcode{modele\_isotropique}}{\emph{I}, \emph{Ib}, \emph{Id}, \emph{beta}, \emph{Rb}, \emph{rhog}}{}
r'`' Petite descrition de la fonction icluant formules...
\begin{quote}\begin{description}
\item[{Parameters}] \leavevmode
\textbf{\texttt{param}} (\href{https://docs.python.org/library/functions.html\#float}{\emph{float}}) -- Ce paramètre est...

\item[{Returns}] \leavevmode
\textbf{param} -- Ce paramètre est...

\item[{Return type}] \leavevmode
\href{https://docs.python.org/library/functions.html\#float}{float}

\end{description}\end{quote}
\paragraph{Notes}

Notes s'il en as.
\paragraph{References}

La référence.

\end{fulllineitems}

\index{modele\_hay\_davis() (in module solar\_mod)}

\begin{fulllineitems}
\phantomsection\label{python:solar_mod.modele_hay_davis}\pysiglinewithargsret{\code{solar\_mod.}\bfcode{modele\_hay\_davis}}{\emph{I}, \emph{Ib}, \emph{Id}, \emph{beta}, \emph{Rb}, \emph{rhog}, \emph{Io}}{}
r'`' Petite descrition de la fonction icluant formules...
\begin{quote}\begin{description}
\item[{Parameters}] \leavevmode
\textbf{\texttt{param}} (\href{https://docs.python.org/library/functions.html\#float}{\emph{float}}) -- Ce paramètre est...

\item[{Returns}] \leavevmode
\textbf{param} -- Ce paramètre est...

\item[{Return type}] \leavevmode
\href{https://docs.python.org/library/functions.html\#float}{float}

\end{description}\end{quote}
\paragraph{Notes}

Notes s'il en as.
\paragraph{References}

La référence.

\end{fulllineitems}

\index{modele\_perez() (in module solar\_mod)}

\begin{fulllineitems}
\phantomsection\label{python:solar_mod.modele_perez}\pysiglinewithargsret{\code{solar\_mod.}\bfcode{modele\_perez}}{\emph{I}, \emph{Ib}, \emph{Id}, \emph{beta}, \emph{Rb}, \emph{rhog}, \emph{Io}, \emph{Ion}, \emph{thez}, \emph{the}}{}
r'`' Petite descrition de la fonction icluant formules...
\begin{quote}\begin{description}
\item[{Parameters}] \leavevmode
\textbf{\texttt{param}} (\href{https://docs.python.org/library/functions.html\#float}{\emph{float}}) -- Ce paramètre est...

\item[{Returns}] \leavevmode
\textbf{param} -- Ce paramètre est...

\item[{Return type}] \leavevmode
\href{https://docs.python.org/library/functions.html\#float}{float}

\end{description}\end{quote}
\paragraph{Notes}

Notes s'il en as.
\paragraph{References}

La référence.

\end{fulllineitems}

\index{snell() (in module solar\_mod)}

\begin{fulllineitems}
\phantomsection\label{python:solar_mod.snell}\pysiglinewithargsret{\code{solar\_mod.}\bfcode{snell}}{\emph{th1}, \emph{nv}, \emph{na=1}}{}
r'`' Petite descrition de la fonction icluant formules...
\begin{quote}\begin{description}
\item[{Parameters}] \leavevmode
\textbf{\texttt{param}} (\href{https://docs.python.org/library/functions.html\#float}{\emph{float}}) -- Ce paramètre est...

\item[{Returns}] \leavevmode
\textbf{param} -- Ce paramètre est...

\item[{Return type}] \leavevmode
\href{https://docs.python.org/library/functions.html\#float}{float}

\end{description}\end{quote}
\paragraph{Notes}

Notes s'il en as.
\paragraph{References}

La référence.

\end{fulllineitems}

\index{r\_coef() (in module solar\_mod)}

\begin{fulllineitems}
\phantomsection\label{python:solar_mod.r_coef}\pysiglinewithargsret{\code{solar\_mod.}\bfcode{r\_coef}}{\emph{th1}, \emph{th2}, \emph{n2=1.526}, \emph{n1=1}}{}
r'`' Petite descrition de la fonction icluant formules...
\begin{quote}\begin{description}
\item[{Parameters}] \leavevmode
\textbf{\texttt{param}} (\href{https://docs.python.org/library/functions.html\#float}{\emph{float}}) -- Ce paramètre est...

\item[{Returns}] \leavevmode
\textbf{param} -- Ce paramètre est...

\item[{Return type}] \leavevmode
\href{https://docs.python.org/library/functions.html\#float}{float}

\end{description}\end{quote}
\paragraph{Notes}

Notes s'il en as.
\paragraph{References}

La référence.

\end{fulllineitems}

\index{Calcul\_coef\_vitre() (in module solar\_mod)}

\begin{fulllineitems}
\phantomsection\label{python:solar_mod.Calcul_coef_vitre}\pysiglinewithargsret{\code{solar\_mod.}\bfcode{Calcul\_coef\_vitre}}{\emph{rpe}, \emph{rpa}, \emph{tau\_al}, \emph{N=1}}{}
r'`' Petite descrition de la fonction icluant formules...
\begin{quote}\begin{description}
\item[{Parameters}] \leavevmode
\textbf{\texttt{param}} (\href{https://docs.python.org/library/functions.html\#float}{\emph{float}}) -- Ce paramètre est...

\item[{Returns}] \leavevmode
\textbf{param} -- Ce paramètre est...

\item[{Return type}] \leavevmode
\href{https://docs.python.org/library/functions.html\#float}{float}

\end{description}\end{quote}
\paragraph{Notes}

Notes s'il en as.
\paragraph{References}

La référence.

\end{fulllineitems}

\index{Calcul\_tau\_al() (in module solar\_mod)}

\begin{fulllineitems}
\phantomsection\label{python:solar_mod.Calcul_tau_al}\pysiglinewithargsret{\code{solar\_mod.}\bfcode{Calcul\_tau\_al}}{\emph{the1}, \emph{alpn}, \emph{KL}, \emph{n2=1.526}, \emph{n1=1}, \emph{N=1}}{}
r'`' Petite descrition de la fonction icluant formules...
\begin{quote}\begin{description}
\item[{Parameters}] \leavevmode
\textbf{\texttt{param}} (\href{https://docs.python.org/library/functions.html\#float}{\emph{float}}) -- Ce paramètre est...

\item[{Returns}] \leavevmode
\textbf{param} -- Ce paramètre est...

\item[{Return type}] \leavevmode
\href{https://docs.python.org/library/functions.html\#float}{float}

\end{description}\end{quote}
\paragraph{Notes}

Notes s'il en as.
\paragraph{References}

La référence.

\end{fulllineitems}

\index{angle\_diffus() (in module solar\_mod)}

\begin{fulllineitems}
\pysiglinewithargsret{\code{solar\_mod.}\bfcode{angle\_diffus}}{\emph{beta}}{}
r'`' Petite descrition de la fonction icluant formules...
\begin{quote}\begin{description}
\item[{Parameters}] \leavevmode
\textbf{\texttt{param}} (\href{https://docs.python.org/library/functions.html\#float}{\emph{float}}) -- Ce paramètre est...

\item[{Returns}] \leavevmode
\textbf{param} -- Ce paramètre est...

\item[{Return type}] \leavevmode
\href{https://docs.python.org/library/functions.html\#float}{float}

\end{description}\end{quote}
\paragraph{Notes}

Notes s'il en as.
\paragraph{References}

La référence.

\end{fulllineitems}

\index{pv\_fct() (in module solar\_mod)}

\begin{fulllineitems}
\phantomsection\label{python:solar_mod.pv_fct}\pysiglinewithargsret{\code{solar\_mod.}\bfcode{pv\_fct}}{\emph{x}, \emph{param}}{}
r'`' Petite descrition de la fonction icluant formules...
\begin{quote}\begin{description}
\item[{Parameters}] \leavevmode
\textbf{\texttt{param}} (\href{https://docs.python.org/library/functions.html\#float}{\emph{float}}) -- Ce paramètre est...

\item[{Returns}] \leavevmode
\textbf{param} -- Ce paramètre est...

\item[{Return type}] \leavevmode
\href{https://docs.python.org/library/functions.html\#float}{float}

\end{description}\end{quote}
\paragraph{Notes}

Notes s'il en as.
\paragraph{References}

La référence.

\end{fulllineitems}

\index{pv\_fct\_194() (in module solar\_mod)}

\begin{fulllineitems}
\phantomsection\label{python:solar_mod.pv_fct_194}\pysiglinewithargsret{\code{solar\_mod.}\bfcode{pv\_fct\_194}}{\emph{x}, \emph{param}}{}
r'`' Petite descrition de la fonction icluant formules...
\begin{quote}\begin{description}
\item[{Parameters}] \leavevmode
\textbf{\texttt{param}} (\href{https://docs.python.org/library/functions.html\#float}{\emph{float}}) -- Ce paramètre est...

\item[{Returns}] \leavevmode
\textbf{param} -- Ce paramètre est...

\item[{Return type}] \leavevmode
\href{https://docs.python.org/library/functions.html\#float}{float}

\end{description}\end{quote}
\paragraph{Notes}

Notes s'il en as.
\paragraph{References}

La référence.

\end{fulllineitems}

\index{pv\_module() (in module solar\_mod)}

\begin{fulllineitems}
\phantomsection\label{python:solar_mod.pv_module}\pysiglinewithargsret{\code{solar\_mod.}\bfcode{pv\_module}}{\emph{xi}, \emph{*param}}{}
r'`' Petite descrition de la fonction icluant formules...
\begin{quote}\begin{description}
\item[{Parameters}] \leavevmode
\textbf{\texttt{param}} (\href{https://docs.python.org/library/functions.html\#float}{\emph{float}}) -- Ce paramètre est...

\item[{Returns}] \leavevmode
\textbf{param} -- Ce paramètre est...

\item[{Return type}] \leavevmode
\href{https://docs.python.org/library/functions.html\#float}{float}

\end{description}\end{quote}
\paragraph{Notes}

Notes s'il en as.
\paragraph{References}

La référence.

\end{fulllineitems}

\index{pv\_module\_194() (in module solar\_mod)}

\begin{fulllineitems}
\phantomsection\label{python:solar_mod.pv_module_194}\pysiglinewithargsret{\code{solar\_mod.}\bfcode{pv\_module\_194}}{\emph{xi}, \emph{*param}}{}
r'`' Petite descrition de la fonction icluant formules...
\begin{quote}\begin{description}
\item[{Parameters}] \leavevmode
\textbf{\texttt{param}} (\href{https://docs.python.org/library/functions.html\#float}{\emph{float}}) -- Ce paramètre est...

\item[{Returns}] \leavevmode
\textbf{param} -- Ce paramètre est...

\item[{Return type}] \leavevmode
\href{https://docs.python.org/library/functions.html\#float}{float}

\end{description}\end{quote}
\paragraph{Notes}

Notes s'il en as.
\paragraph{References}

La référence.

\end{fulllineitems}

\index{I\_pvV() (in module solar\_mod)}

\begin{fulllineitems}
\phantomsection\label{python:solar_mod.I_pvV}\pysiglinewithargsret{\code{solar\_mod.}\bfcode{I\_pvV}}{\emph{x}, \emph{V}, \emph{S=1000.0}, \emph{T=298.15}}{}
r'`' Petite descrition de la fonction icluant formules...
\begin{quote}\begin{description}
\item[{Parameters}] \leavevmode
\textbf{\texttt{param}} (\href{https://docs.python.org/library/functions.html\#float}{\emph{float}}) -- Ce paramètre est...

\item[{Returns}] \leavevmode
\textbf{param} -- Ce paramètre est...

\item[{Return type}] \leavevmode
\href{https://docs.python.org/library/functions.html\#float}{float}

\end{description}\end{quote}
\paragraph{Notes}

Notes s'il en as.
\paragraph{References}

La référence.

\end{fulllineitems}

\index{I\_pvR() (in module solar\_mod)}

\begin{fulllineitems}
\phantomsection\label{python:solar_mod.I_pvR}\pysiglinewithargsret{\code{solar\_mod.}\bfcode{I\_pvR}}{\emph{x}, \emph{R}, \emph{S=1000.0}, \emph{T=298.15}}{}
r'`' Petite descrition de la fonction icluant formules...
\begin{quote}\begin{description}
\item[{Parameters}] \leavevmode
\textbf{\texttt{param}} (\href{https://docs.python.org/library/functions.html\#float}{\emph{float}}) -- Ce paramètre est...

\item[{Returns}] \leavevmode
\begin{itemize}
\item {} 
\textbf{S} (\emph{float}) -- (Default value = 1000.0)

\item {} 
\textbf{T} (\emph{float}) -- (Default value = 25.0 + 273.15)

\end{itemize}


\end{description}\end{quote}
\paragraph{Notes}

Notes s'il en as.
\paragraph{References}

La référence.

\end{fulllineitems}

\index{IV\_pv\_peak() (in module solar\_mod)}

\begin{fulllineitems}
\phantomsection\label{python:solar_mod.IV_pv_peak}\pysiglinewithargsret{\code{solar\_mod.}\bfcode{IV\_pv\_peak}}{\emph{x}, \emph{S=1000.0}, \emph{T=298.15}}{}
r'`' Petite descrition de la fonction icluant formules...
\begin{quote}\begin{description}
\item[{Parameters}] \leavevmode
\textbf{\texttt{param}} (\href{https://docs.python.org/library/functions.html\#float}{\emph{float}}) -- Ce paramètre est...

\item[{Returns}] \leavevmode
\textbf{param} -- Ce paramètre est...

\item[{Return type}] \leavevmode
\href{https://docs.python.org/library/functions.html\#float}{float}

\end{description}\end{quote}
\paragraph{Notes}

Notes s'il en as.
\paragraph{References}

La référence.

\end{fulllineitems}

\index{cherche\_index() (in module solar\_mod)}

\begin{fulllineitems}
\phantomsection\label{python:solar_mod.cherche_index}\pysiglinewithargsret{\code{solar\_mod.}\bfcode{cherche\_index}}{\emph{xi}, \emph{x}}{}
r'`' Petite descrition de la fonction icluant formules...
\begin{quote}\begin{description}
\item[{Parameters}] \leavevmode
\textbf{\texttt{param}} (\href{https://docs.python.org/library/functions.html\#float}{\emph{float}}) -- Ce paramètre est...

\item[{Returns}] \leavevmode
\textbf{param} -- Ce paramètre est...

\item[{Return type}] \leavevmode
\href{https://docs.python.org/library/functions.html\#float}{float}

\end{description}\end{quote}
\paragraph{Notes}

Notes s'il en as.
\paragraph{References}

La référence.

\end{fulllineitems}



\section{data\_mod}
\label{python:data-mod}
Faire une module pour le traitement des données météo


\section{tools\_mod}
\label{python:tools-mod}
Faire une module personnel pour les outils


\chapter{Sphinx}
\label{sphinx::doc}\label{sphinx:sphinx}
\begin{notice}{note}{Todo}

Documenter toutes les fonctions nécessaires.
\end{notice}

Pour nettoyer et générer le fichier html de la documentation:

\begin{Verbatim}[commandchars=\\\{\}]
cd C:\PYGZbs{}Users\PYGZbs{}Antoine\PYGZbs{}Documents\PYGZbs{}GitHub\PYGZbs{}Solar\PYGZus{}mod\PYGZbs{}docs
make clean
make html
\end{Verbatim}


\section{Mise en page}
\label{sphinx:mise-en-page}

\section{Titre de niveau 1}
\label{sphinx:titre-de-niveau-1}

\subsection{Titre de niveau 2}
\label{sphinx:titre-de-niveau-2}

\subsubsection{Titre de niveau 3}
\label{sphinx:titre-de-niveau-3}

\paragraph{Titre de niveau 4}
\label{sphinx:titre-de-niveau-4}

\subparagraph{Titre de niveau 5}
\label{sphinx:titre-de-niveau-5}

\subsection{Insérer une note}
\label{sphinx:inserer-une-note}
\begin{notice}{note}{Note:}
J'espère que cela vous parle un peu plus que la présentation présente.
\end{notice}


\subsection{Insérer un todo}
\label{sphinx:inserer-un-todo}
\begin{notice}{note}{Todo}

Liste de choses à faire
\end{notice}


\section{Mathématique}
\label{sphinx:mathematique}

\subsection{Formule mathématique du type LaTex}
\label{sphinx:formule-mathematique-du-type-latex}\begin{gather}
\begin{split}\sum_{a}^{b} x_n = c_1 + c_2^3 + \cos(\pi)\end{split}\notag
\end{gather}

\section{Graphique}
\label{sphinx:graphique}

\subsection{Insérer une image du static}
\label{sphinx:inserer-une-image-du-static}
{\hfill\includegraphics{python-powered.png}}
\begin{figure}[htbp]
\centering

\includegraphics{ETS.png}
\end{figure}

figure are like images but with a caption and whatever else youwish to add


\section{Python}
\label{sphinx:python}

\section{Autres}
\label{sphinx:autres}

\subsection{Insérer un lien}
\label{sphinx:inserer-un-lien}
\href{http://www.python.org/}{Python}


\chapter{Devoirs}
\label{devoir::doc}\label{devoir:devoirs}
\begin{notice}{note}{Todo}

Page d'info relative au cours
\end{notice}


\section{Devoir \#2}
\label{devoir:devoir-2}

\subsection{Question 1}
\label{devoir:question-1}

\subsubsection{1.}
\label{devoir:id1}

\chapter{Projet}
\label{projet::doc}\label{projet:projet}
\begin{notice}{note}{Todo}

Page d'info relative au projet
\end{notice}


\chapter{trnsys}
\label{trnsys::doc}\label{trnsys:trnsys}
\begin{notice}{note}{Todo}

Page d'info relative a trnsys
\end{notice}


\chapter{Indices and tables}
\label{index:indices-and-tables}\begin{itemize}
\item {} 
\DUspan{xref,std,std-ref}{genindex}

\item {} 
\DUspan{xref,std,std-ref}{modindex}

\item {} 
\DUspan{xref,std,std-ref}{search}

\end{itemize}


\renewcommand{\indexname}{Python Module Index}
\begin{theindex}
\def\bigletter#1{{\Large\sffamily#1}\nopagebreak\vspace{1mm}}
\bigletter{s}
\item {\texttt{solar\_mod}}, \pageref{python:module-solar_mod}
\end{theindex}

\renewcommand{\indexname}{Index}
\printindex
\end{document}
